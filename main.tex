%\PassOptionsToPackage{svgnames}{xcolor} %farbige blöcke
\documentclass[
%DIV = calc, %automatische Berechnung vonsseitenrändern
10pt,
a4paper,
oneside,
bibtotoc, 
liststotoc, 
headsepline, 
smallheadings, 
openright, 
fleqn,
appendixprefix,
draft,
BCOR5mm]
{scrbook}

%%Pfadprüfung%%
\IfFileExists{../../_preamble/check_file.tex}% Prüfung aus welcher Datei der Aufruf stattfindet
{%
\providecommand{\myPath}{../../}% file exists=true: befinde mich im Unterverzeichnis
}%
{%
\providecommand{\myPath}{}% file exists=false: befinde mich im RootVerzeichnis
}%
%%Pfadprüfung%%

%%Einbinden globaler Pakete und Paketeinstellungen%%
%//Allgemeine Pakete\\%
\usepackage[utf8]{inputenc}				% OK, Korrektes Encoding in Latex-Schrift, 							check: 25.08.18
\usepackage[ngerman]{babel}				% OK, Korrekte Schreibweise und Treenungsregelung					check: 25.08.18
\usepackage[T1]{fontenc}				% OK, Korrektes Font-Encoding			 							check: 25.08.18
\usepackage{amsmath} 					% OK, Mathematische Symbole und Schriften, 							check: 25.08.18
\usepackage{amsfonts} 					% OK, Mathematische Symbole und Schriften, 							check: 25.08.18
\usepackage{amssymb} 					% OK, Mathematische Symbole und Schriften, 							check: 25.08.18
\usepackage{amsthm} 					% OK, Typesetting Theoreme, 										check: 25.08.18
\usepackage{exscale}  					% OK, Skalierung von mathematischen Symbolen, 						check: 25.08.18
\usepackage{amstext} 					% OK, Text in mathematischer Umgebung, 								check: 25.08.18 [enthalten in amsmath]
\usepackage{verbatim} 					% OK, Funktion zum Schreiben von unformatiertem Text, 				check: 25.08.18
\usepackage{float} 						% OK, Ermöglicht das Erstellen von eigenen Float Umgebungen, 		check: 25.08.18
%\usepackage{makeidx}					% OK, Erstellen von Indizes zur Anlegung eines Glossars, 			check: 25.08.18
\usepackage{standalone} 				% OK, Kompilierung als eigenes Dok und Einbindung über Main,		check: 25.08.18
%\usepackage{import} 					% OK, Übernahme von relativen Verweisen in \inputs,					check: 25.08.18


%//Syntaktische Formatierung\\%
\usepackage{scrlayer-scrpage}			% OK, Definition und Manipulation von Kopf-, Fußzeile,				check: 25.08.18
\usepackage[onehalfspacing]{setspace}	% OK, Einstellen Zeilenabstand (onehalfspacing, doublespacing)		check: 25.08.18
\usepackage[							% OK, Definition des Seitenlayouts,									check: 25.08.18
includeheadfoot,						%
left=3cm,right=2cm,top=2cm,bottom=2cm,	%
showframe								%
]{geometry}								%
\usepackage[section]{placeins} 			% OK, Verhindert das Wandern von floats in eine andere section, 	check: 25.08.18
\usepackage{microtype} 					% OK, Abstand zwischen Zeichen anpassen zur Darstellung,			check: 25.08.18
\usepackage{lmodern}					% OK, Latin Modern font,											check: 25.08.18


%//Bilder und Tabellen\\%
\usepackage{xcolor} 					% OK, Definieren und Nutzen von versch. Farben						check: 25.08.18
\usepackage{graphicx}					% OK, Bereitstellen von \includegraphics							check: 25.08.18
\usepackage{array} 						% OK, Erstellen eigener Columntypen in Tabellenumgebungen			check: 25.08.18
\usepackage{rotating} 					% OK, Rotieren von floats in boxen (Tabellen, Bilder)				check: 25.08.18
\usepackage{adjustbox}					% OK, Ähnlich wie resizebox für normale Textumgebung,				check: 25.08.18
\usepackage{booktabs}					% OK, Schönere Tabellen ohne vertikale Linien \toprule etc.			check: 25.08.18
\usepackage{tabularx}					% OK, Weiterer Spaltentyp passt Tabellenbreite  automatisch an		check: 25.08.18
\usepackage{multirow}					% OK, Zellenspannung über mehrere Zeilen							check: 25.08.18
%\usepackage{makecell}					% OK, Tabellenlayout (Tabaellenheader) ähnlich \multirow			check: 25.08.18
\usepackage{tablefootnote}				% OK, Fußnoten in Tabellen (\footnote funktioniert nicht)			check: 25.08.18
\usepackage{subcaption} 				% OK, Darstellung von Tabellenumgebungen aus mehreren (auch Figure)	check: 25.08.18


%//Diverses\\%
\usepackage{paralist} 					% OK, Aufzählung wie \itemize \compactenum (nochmal anschauen)		check: 25.08.18
\usepackage[withpage]{acronym}			% OK, Genutze Abkürzungen werden das erste mal ausgeschrieben		check: 25.08.18
\usepackage{fancybox}					% OK, Verschiedene Varianten von \fbox zur Umrahmung von Text		check: 25.08.18
\usepackage[most]{tcolorbox}			% OK, Colorierte und umrahmte Textboxen wie \fbox für Theoreme		check: 25.08.18
\usepackage{layouts}					% OK, Darstellen von Dokumenteninformationen,						check: 25.08.18
%\usepackage[textsize=tiny]{todonotes}	% OK, Markierung und Erstellung von todo-Anmerkungen				check: 25.08.18
%\usepackage[timer]{regstats}			% Keine Funktion?, Erhebung von Statistiken (Kompiluierungszeit)	check: 25.08.18
%\usepackage[l2tabu, orthodox]{nag}		% Keine Funktion?, Prüfung nach veralteten Paketen					check: 25.08.18
%\usepackage{mylatexformat}


%//Theoreme\\%
%\usepackage{ntheorem} 					% OK, Darstellung von Theoremen										check: 25.08.18
%\usepackage{thm­tools} 				% OK, Darstellung von Theoremen <-									check: 25.08.18
\newtcbtheorem[number within=section, list inside=def]{mydef}{Definition}%
{enhanced, colback=white,colframe=\fillTwo,fonttitle=\bfseries, description delimiters={\flqq}{\frqq}, description font=\mdseries\ttfamily, boxrule=0.4pt,toprule=3pt, boxsep=3pt,arc=0mm, list text={test}, drop shadow}{def}


%//Zitate und Links\\%
\usepackage{csquotes}					% OK, Korrekte Darstellung von "" je nach Land						check: 25.08.18
\usepackage[]{hyperref}					% OK, Cross-Referencing und hypermarks im Dokument					check: 25.08.18
\hypersetup{
bookmarksopen=true,
%pdfpagelabels=true,					%Option `pdfpagelabels' has already been used,( hyperref) setting the option has no effect on input line 74
plainpages=false,
colorlinks=true,
citecolor=blue,
linkcolor=blue,
urlcolor=blue}

%//BibLatex\\%
\usepackage[backend=biber,				% OK, Neue Version von BibTex zur Darstellung Literaturverzeichnis	check: 25.08.18 
style=authoryear, %% Zitierstil
natbib=true, %% Bereitstellen von natbib-kompatiblen Zitierkommandos
hyperref=true, %% hyperref-Paket verwenden, um Links zu erstellen
maxcitenames=1,
uniquelist=false,
bibstyle=numeric,
url=false,
doi=false
]{biblatex}
%\addbibresource{bibfile.bib} 			% Einbinden der bib-Datei
%\DefineBibliographyStrings{ngerman}{andothers={et al.}} % et al. statt u.a.
%\renewcommand*{\nameyeardelim}{\addcomma\space} % Macht im Zitat ein Komma vor das Jahr

%%Font etc.%%
\usepackage{lmodern}					% OK, Latin Modern font,											check: 25.08.18
\usepackage{xcolor} 					% OK, Definieren und Nutzen von versch. Farben						check: 25.08.18
\usepackage{graphicx}					% OK, Bereitstellen von \includegraphics							check: 25.08.18

%%Forest%%
\usepackage{forest}						% OK, Baumdarstellung aus dem linguistischen Bereich				check: 25.08.18
\useforestlibrary{edges}

%%Venndiagramm%%
\usepackage{venndiagram}				% OK, Definieren und Darstellen von Venndiagrammen					check: 25.08.18

%%Tabellen%%
\usepackage{booktabs}					% OK, Schönere Tabellen ohne vertikale Linien \toprule etc.			check: 25.08.18
\usepackage{tabularx}					% OK, Weiterer Spaltentyp passt Tabellenbreite  automatisch an		check: 25.08.18
\usepackage{multirow}					% OK, Zellenspannung über mehrere Zeilen							check: 25.08.18
%\usepackage{makecell}					% OK, Tabellenlayout (Tabaellenheader) ähnlich \multirow			check: 25.08.18
\usepackage{tablefootnote}				% OK, Fußnoten in Tabellen (\footnote funktioniert nicht)			check: 25.08.18
\usepackage{array} 						% OK, Erstellen eigener Columntypen in Tabellenumgebungen			check: 25.08.18

%%Grafiken und Plots%%
\usepackage{tikz} 						% OK, Natives zeichnen in Latex ,									check: 25.08.18
\usepackage{tikz-cd} 					% OK, Erstellen von kommutativen Diagrammen in Tikz,				check: 25.08.18
\usepackage{pgfplots}					% OK, Plotten von Daten,											check: 25.08.18
\pgfplotsset{compat=newest}				% OK, Einstellen der Kompatibilitätsversion,						check: 25.08.18
\usepackage{pgfplotstable}				% OK, Plotten und schreiben von Daten in Tabellen,					check: 25.08.18
\usepackage{pgfcalendar} 				% OK, Umrechnen von Datumskoordinaten,								check: 25.08.18
\usepgfplotslibrary{dateplot}			% OK, Plotten von Datumskoordinaten,								check: 25.08.18
\usepgfplotslibrary{units}				% OK, Darstellen von Einheiten als Achsenlabel,						check: 25.08.18
%//Tikzbibliotheken\\%
\usetikzlibrary{						% Bibliotheken zur direkten Einbindung in TIKZEdt
arrows, 
fit,
shapes.geometric, 
matrix,
calc,
decorations.markings,
decorations.pathreplacing,
decorations.pathmorphing,
backgrounds,
shadings, 
shadows,
positioning,
mindmap,
trees,
datavisualization
}
%%define tikz-stlyes here, colours etc.
\tikzset{
block_phantom/.style={block_normal, draw=red, fill=none},
block_phantom/.style={block_normal, draw=blue, fill=none}
}
%\tikzstyle{block_phantom}=[block_normal, draw=red, fill=none]
\pgfplotsset{
  global_appearance/.style={
  grid_appearance,
  scale only axis,
  width= .75\textwidth,
  %width=12.5cm, % 13,5
  height=.4\textwidth
  }
}

\pgfplotsset{
  grid_appearance/.style={
   grid,
   grid style = {loosely dotted , thin},
  }
}

\pgfplotsset{
 %every axis plot/.append style={no markers},
  every pin/.append style={font=\footnotesize },
  every mark/.append style={scale=3},
  legend style=
  {
  font=\tiny, 
  %draw=none, 
  cells={align=left}, 
  at={(0.5,1.00)},
  anchor=north,
 % legend pos= north west
  },
  legend columns = 2,
  label style={font=\scriptsize},
  tick label style={font=\tiny},
}

\pgfplotsset{
  tuftle_like_axes/.style={
  thin,  %vorher semithick
  tick style={major tick length=4pt,thin,black}, %vorher semithick
  separate axis lines,
  axis x line*=bottom,
  axis x line shift=10pt,
  xlabel shift=5pt,
  axis y line*=left,
  axis y line shift=10pt,
  tick align = outside,
  ylabel shift=2pt,
  enlarge y limits=false,
  enlarge x limits=false,
  }
}

%%Tufte-Design%%
% \makeatletter
% \pgfplotsset{
  % tufte axes/.style =
    % {
      % after end axis/.code =
        % {
          % \draw ({rel axis cs:0,0} -| {axis cs:\pgfplots@data@xmin,0}) -- ({rel axis cs:0,0}  -| {axis cs:\pgfplots@data@xmax,0});
          % \draw ({rel axis cs:0,0} |- {axis cs:0,\pgfplots@data@ymin}) -- ({rel axis cs:0,0}  |-{axis cs:0,\pgfplots@data@ymax});
        % },
      % axis line style = {draw = none},
      % tick align      = outside,
      % tick pos        = left
    % }
% }
% \makeatother
%//Commands\\%
\newcommand{\fillOne}{green!35!black}						% Farbenneudefinition	
\newcommand{\fillTwo}{black!70}								% Farbenneudefinition	
\newcommand{\fillThree}{blue!30!}							% Farbenneudefinition	
\newcommand{\fillFour}{black!10!blue}						% Farbenneudefinition	

\newlength{\freewidth}										% Zur Berechnung noch freien Platzes bei prozentualer Angabe der Spaltenbreite in Tabellen

%//Settings\\%
\setcounter{secnumdepth}{3}									% Sektionstiefe definieren zur Darstellung im Inhaltsverzeichnis		
\setcounter{tocdepth}{4}									% Tiefe definieren zur Darstellung im Inhaltsverzeichnis

\renewcommand{\topfraction}{0.85} 							% Anteil einer Seite, die von Floats belegt sein darf (default 0.7)
\renewcommand{\textfraction}{0.1} 							% Anteil einer Seite, die mindestens Text sein muss (default 0.2)
\renewcommand{\floatpagefraction}{0.75} 					% Anteil einer Seite, die ein Float einnehmen darf, ohne auf die nächste Seite gesetzt zu werden (default 0.5)	

\newtheorem{theorem}{Satz}									% Theorem Satz
\newtheorem{definition}{Definition}							% Theorem Definition <-
\newtheorem{lemma}{Lemma}									% Theorem Lemma

\newcolumntype{L}[1]{>{\raggedright\arraybackslash}p{#1}} 	% Linksbündig mit Breitenangabe
\newcolumntype{C}[1]{>{\centering\arraybackslash}p{#1}} 	% Zentriert mit Breitenangabe
\newcolumntype{R}[1]{>{\raggedleft\arraybackslash}p{#1}} 	% Rechtsbündig mit Breitenangabe
%\newcolumntype{C}[1]{>{\centering\arraybackslash}m{#1}} 	% Vertikale zentrierung

\captionsetup[figure]{										% Captionsetup für Figure-Umgebung
%skip=5cm,
position=below,
justification=centering,
font=small
}
\captionsetup[table]{										% Captionsetup für Tabellen-Umgebung
%skip=5cm,
position=below,
justification=centering,
font=small}
\setlength\abovecaptionskip{10pt}
\setlength\belowcaptionskip{10pt}





%%Einbinden globaler Pakete und Paketeinstellungen%%




\begin{document}

%% Ausgabe Dokumenteneigenschaften
%% display some document properties

\edef\defaultfont{\fontname\font}  % Speichert die Default Font inkl. Schriftgrad
\edef\defaultsize{\csname f@size\endcsname}  % Speichert die Default Font inkl. Schriftgrad
\newcommand{\currentfont}{\fontname\font}
\newcommand{\currentsize}{\csname f@size\endcsname}

\begin{table}[]
\footnotesize
\centering
\setlength{\freewidth}{\dimexpr\textwidth-6\tabcolsep}
\begin{tabular}{L{.8\freewidth}L{.1\freewidth}L{.10\freewidth}}   % 100% without resizbox
\toprule
Feature & Value \\
\midrule
PaperWidth in [cm]& \printinunitsof{cm}\prntlen{\paperwidth}\\
PaperHeight in [cm]& \printinunitsof{cm}\prntlen{\paperheight}\\
TextWidth in [cm] &  \printinunitsof{cm}\prntlen{\textwidth}  \\
TextHeight in [cm] & \printinunitsof{cm}\prntlen{\textheight}\\
\midrule
MarginParSep in [cm] & \printinunitsof{cm}\prntlen{\marginparsep}\\
MarginParWidth in [cm] & \printinunitsof{cm}\prntlen{\marginparwidth}\\
FootSkip in [cm]& \printinunitsof{cm}\prntlen{\footskip}\\
\midrule
Default Font &  \defaultfont \\
Default Size   & \defaultsize \\
Current Font &  \currentfont \\
Current Size &  \currentsize \\
\midrule
Colour1 & \fillOne & \begin{tikzpicture} \draw[fill=\fillOne, draw=none] (0,0) rectangle (1,.2); \end{tikzpicture}\\
Colour2 & \fillTwo & \begin{tikzpicture} \draw[fill=\fillTwo, draw=none] (0,0) rectangle (1,.2); \end{tikzpicture}\\
Colour3 & \fillThree & \begin{tikzpicture} \draw[fill=\fillThree, draw=none] (0,0) rectangle (1,.2); \end{tikzpicture}\\
Colour4 & \fillFour & \begin{tikzpicture} \draw[fill=\fillFour, draw=none] (0,0) rectangle (1,.2); \end{tikzpicture}\\
\bottomrule 
\end{tabular}
\end{table}

%% Ausgabe Dokumenteneigenschaften


\begin{onehalfspace} %Start 1,5-facher Zeilenabstand
\pagestyle{empty}
\pagenumbering{alph}
\input{\myPath_text/0_pre_mainmatter/01_title.tex} \cleardoublepage
\input{\myPath_text/0_pre_mainmatter/02_erklaerung.tex} \cleardoublepage
\input{\myPath_text/0_pre_mainmatter/03_abstract.tex} \cleardoublepage
\input{\myPath_text/0_pre_mainmatter/04_vorwort.tex} \cleardoublepage
\pagestyle{scrheadings}
\frontmatter
\tableofcontents \cleardoublepage
%\addcontentsline{toc}{chapter}{\listfigurename}
 \listoffigures \cleardoublepage
%\listofalgorithms \cleardoublepage
\listoftables \cleardoublepage
%\tcblistof[\chapter]{def}{Definitionsverzeichnis}\cleardoublepage
%\listoftodos \cleardoublepage
\input{\myPath_text/0_pre_mainmatter/05_abkuerzung.tex} \cleardoublepage

%//Ausprobieren\\%
%\newfloat{ergebnisse}{thb}{loh}[chapter] % Liste der detaillierten Ergebnisse
%\floatname{ergebnisse}{Ergebnisse}
%\listof{ergebnisse}{Liste der detaillierten Ergebnisse}
%\cleardoublepage
%//Ausprobieren\\%


\mainmatter % der eigentliche Text folgt hier
\input{\myPath_text/1_mainmatter/010_einleitung.tex}
\chapter{Theoretische Grundlagen und Definitionen}

Das folgende Kapitel dient der Bereitstellung von Grundlagenwissen zu in der Arbeit verwendeten Konzepten bereit. Detaillierte Ausführungen finden sich in den im Literaturverzeichnis aufgeführten Werken wieder. Zusätzlich werden Definitionen festgelgt.
Dient der theroethischen Fundiuerng. Dabeiw erden grundlegende Konzepte zu den Themen Kreativität, experimentelle Forschung, Wirtschaftsinformatik und der Geschäftsmodelltheorie gegeben.


\begin{mydef}{Experiment}{def_exp}
  Im Zuge dieser Arbeit impliziert der Begriff „Wirtschaftsinformatik“ die Betrachtung (konstruierter) IT-Artefakte ausschließlich in Form softwareseitiger Anwendungssysteme.
  This theorem is numbered with  \ref{def:def_exp} and is given on page \pageref{def:def_exp}.
\end{mydef}



%%Einbindung der Grafiken rotiert %% 
 
 \begin{figure}[h]
 \centering
 \resizebox{\textwidth}{!}{%
 \IfFileExists{../../_preamble/check_file.tex}% prüfen aus welcher datei der aufruf stattfindet
{%
\providecommand{\myPath}{../../}% file exists=true: befinde mich im unterverzeichnis
}%
{%
\providecommand{\myPath}{}% file exists=false: befinde mich im root_verzeichnis
}%
\input{\myPath_preamble/class_declaration_sa.tex}%% läd standalone-klasse mit tikz-argument
%//Tikzbibliotheken\\%
\usetikzlibrary{						% Bibliotheken zur direkten Einbindung in TIKZEdt
arrows, 
fit,
shapes.geometric, 
matrix,
calc,
decorations.markings,
decorations.pathreplacing,
decorations.pathmorphing,
backgrounds,
shadings, 
shadows,
positioning,
mindmap,
trees,
datavisualization
}%% diverse tikz-bibliotheken
%%Font etc.%%
\usepackage{lmodern}					% OK, Latin Modern font,											check: 25.08.18
\usepackage{xcolor} 					% OK, Definieren und Nutzen von versch. Farben						check: 25.08.18
\usepackage{graphicx}					% OK, Bereitstellen von \includegraphics							check: 25.08.18

%%Forest%%
\usepackage{forest}						% OK, Baumdarstellung aus dem linguistischen Bereich				check: 25.08.18
\useforestlibrary{edges}

%%Venndiagramm%%
\usepackage{venndiagram}				% OK, Definieren und Darstellen von Venndiagrammen					check: 25.08.18

%%Tabellen%%
\usepackage{booktabs}					% OK, Schönere Tabellen ohne vertikale Linien \toprule etc.			check: 25.08.18
\usepackage{tabularx}					% OK, Weiterer Spaltentyp passt Tabellenbreite  automatisch an		check: 25.08.18
\usepackage{multirow}					% OK, Zellenspannung über mehrere Zeilen							check: 25.08.18
%\usepackage{makecell}					% OK, Tabellenlayout (Tabaellenheader) ähnlich \multirow			check: 25.08.18
\usepackage{tablefootnote}				% OK, Fußnoten in Tabellen (\footnote funktioniert nicht)			check: 25.08.18
\usepackage{array} 						% OK, Erstellen eigener Columntypen in Tabellenumgebungen			check: 25.08.18

%%Grafiken und Plots%%
\usepackage{tikz} 						% OK, Natives zeichnen in Latex ,									check: 25.08.18
\usepackage{tikz-cd} 					% OK, Erstellen von kommutativen Diagrammen in Tikz,				check: 25.08.18
\usepackage{pgfplots}					% OK, Plotten von Daten,											check: 25.08.18
\pgfplotsset{compat=newest}				% OK, Einstellen der Kompatibilitätsversion,						check: 25.08.18
\usepackage{pgfplotstable}				% OK, Plotten und schreiben von Daten in Tabellen,					check: 25.08.18
\usepackage{pgfcalendar} 				% OK, Umrechnen von Datumskoordinaten,								check: 25.08.18
\usepgfplotslibrary{dateplot}			% OK, Plotten von Datumskoordinaten,								check: 25.08.18
\usepgfplotslibrary{units}				% OK, Darstellen von Einheiten als Achsenlabel,						check: 25.08.18%% nur laden wenn weitere graphic pakete benötigt werden (tabellen, pgfplot,...)
%%define tikz-stlyes here, colours etc.
\tikzset{
block_phantom/.style={block_normal, draw=red, fill=none},
block_phantom/.style={block_normal, draw=blue, fill=none}
}
%\tikzstyle{block_phantom}=[block_normal, draw=red, fill=none]%% tikz-styles, farben etc.
\begin{document}
\begin{forest}
for tree={grow=south},forked edges,
[Experimente zum Thema Kreativität in der Wirtschatfsinformatik
    [ Veröffentlichung
        [<1990]
        [<2000]
        [<2010]
        [>2010]
    ]
    [Aufgabenart
        [Generierung
            [Probandenrelevant]
            [Nicht Probandenrelevant]
        ]
        [Selektion
            [Probandenrelevant]
            [Nicht Probandenrelevant]
        ]
    ]
    [,phantom
        [,phantom
            [,phantom
                [test
                    [test1] [test2 [test3]]
                ]
            ]
        ]
    ]
    [Teilnehmermotivation
        [Bezahlung
            [>10 Euro]
            [<=10 Euro]
        ]
        [Leistungspunkte]
        [Teilnehmerrelevanz]
    ]
    [Kontrolle von Störgrößen
    [4
            [Gruppe6]
        ]
        [Gruppe5]
    ]
    [Gruppe 4
    [4
            [Gruppe6]
        ]
        [Gruppe5 [Gruppe4[Gruppe8[7[0[Gruppe4]]]]]]
    ]
    [Gruppe 5
    [Gruppe4
            [Gruppe6]
        ]
        [5]
    ]
    [Gruppe 6
    [Gruppe4
            [Gruppe6]
        ]
        [Gruppe5]
    ]
]
\end{forest}

\end{document}
%
 }
 \caption{}
 \end{figure}

 
\begin{sidewaysfigure}
    \centering
    \resizebox{\textwidth}{!}{%
    \IfFileExists{../../_preamble/check_file.tex}% prüfen aus welcher datei der aufruf stattfindet
{%
\providecommand{\myPath}{../../}% file exists=true: befinde mich im unterverzeichnis
}%
{%
\providecommand{\myPath}{}% file exists=false: befinde mich im root_verzeichnis
}%
\input{\myPath_preamble/class_declaration_sa.tex}%% läd standalone-klasse mit tikz-argument
%//Tikzbibliotheken\\%
\usetikzlibrary{						% Bibliotheken zur direkten Einbindung in TIKZEdt
arrows, 
fit,
shapes.geometric, 
matrix,
calc,
decorations.markings,
decorations.pathreplacing,
decorations.pathmorphing,
backgrounds,
shadings, 
shadows,
positioning,
mindmap,
trees,
datavisualization
}%% diverse tikz-bibliotheken
%%Font etc.%%
\usepackage{lmodern}					% OK, Latin Modern font,											check: 25.08.18
\usepackage{xcolor} 					% OK, Definieren und Nutzen von versch. Farben						check: 25.08.18
\usepackage{graphicx}					% OK, Bereitstellen von \includegraphics							check: 25.08.18

%%Forest%%
\usepackage{forest}						% OK, Baumdarstellung aus dem linguistischen Bereich				check: 25.08.18
\useforestlibrary{edges}

%%Venndiagramm%%
\usepackage{venndiagram}				% OK, Definieren und Darstellen von Venndiagrammen					check: 25.08.18

%%Tabellen%%
\usepackage{booktabs}					% OK, Schönere Tabellen ohne vertikale Linien \toprule etc.			check: 25.08.18
\usepackage{tabularx}					% OK, Weiterer Spaltentyp passt Tabellenbreite  automatisch an		check: 25.08.18
\usepackage{multirow}					% OK, Zellenspannung über mehrere Zeilen							check: 25.08.18
%\usepackage{makecell}					% OK, Tabellenlayout (Tabaellenheader) ähnlich \multirow			check: 25.08.18
\usepackage{tablefootnote}				% OK, Fußnoten in Tabellen (\footnote funktioniert nicht)			check: 25.08.18
\usepackage{array} 						% OK, Erstellen eigener Columntypen in Tabellenumgebungen			check: 25.08.18

%%Grafiken und Plots%%
\usepackage{tikz} 						% OK, Natives zeichnen in Latex ,									check: 25.08.18
\usepackage{tikz-cd} 					% OK, Erstellen von kommutativen Diagrammen in Tikz,				check: 25.08.18
\usepackage{pgfplots}					% OK, Plotten von Daten,											check: 25.08.18
\pgfplotsset{compat=newest}				% OK, Einstellen der Kompatibilitätsversion,						check: 25.08.18
\usepackage{pgfplotstable}				% OK, Plotten und schreiben von Daten in Tabellen,					check: 25.08.18
\usepackage{pgfcalendar} 				% OK, Umrechnen von Datumskoordinaten,								check: 25.08.18
\usepgfplotslibrary{dateplot}			% OK, Plotten von Datumskoordinaten,								check: 25.08.18
\usepgfplotslibrary{units}				% OK, Darstellen von Einheiten als Achsenlabel,						check: 25.08.18%% nur laden wenn weitere graphic pakete benötigt werden (tabellen, pgfplot,...)
%%define tikz-stlyes here, colours etc.
\tikzset{
block_phantom/.style={block_normal, draw=red, fill=none},
block_phantom/.style={block_normal, draw=blue, fill=none}
}
%\tikzstyle{block_phantom}=[block_normal, draw=red, fill=none]%% tikz-styles, farben etc.
\begin{document}
\begin{forest}
for tree={grow=south},forked edges,
[Experimente zum Thema Kreativität in der Wirtschatfsinformatik
    [ Veröffentlichung
        [<1990]
        [<2000]
        [<2010]
        [>2010]
    ]
    [Aufgabenart
        [Generierung
            [Probandenrelevant]
            [Nicht Probandenrelevant]
        ]
        [Selektion
            [Probandenrelevant]
            [Nicht Probandenrelevant]
        ]
    ]
    [,phantom
        [,phantom
            [,phantom
                [test
                    [test1] [test2 [test3]]
                ]
            ]
        ]
    ]
    [Teilnehmermotivation
        [Bezahlung
            [>10 Euro]
            [<=10 Euro]
        ]
        [Leistungspunkte]
        [Teilnehmerrelevanz]
    ]
    [Kontrolle von Störgrößen
    [4
            [Gruppe6]
        ]
        [Gruppe5]
    ]
    [Gruppe 4
    [4
            [Gruppe6]
        ]
        [Gruppe5 [Gruppe4[Gruppe8[7[0[Gruppe4]]]]]]
    ]
    [Gruppe 5
    [Gruppe4
            [Gruppe6]
        ]
        [5]
    ]
    [Gruppe 6
    [Gruppe4
            [Gruppe6]
        ]
        [Gruppe5]
    ]
]
\end{forest}

\end{document}
%
    }
    \caption{ Here is a caption of the figure which is so long that 
      it has to be wrapped over multiple lines, but should 
      not exceed the width (height after the rotation) of the image.}
    \label{fig:awesome_image}
\end{sidewaysfigure}

 
\begin{figure}[ht]
  \begin{adjustbox}{addcode={\begin{minipage}{\width}}{\caption{%
      Here is a caption of the figure which is so long that 
      it has to be wrapped over multiple lines, but should 
      not exceed the width (height after the rotation) of the image.
      }\end{minipage}},rotate=90,center}
      \resizebox{\textheight}{!}{%
      \IfFileExists{../../_preamble/check_file.tex}% prüfen aus welcher datei der aufruf stattfindet
{%
\providecommand{\myPath}{../../}% file exists=true: befinde mich im unterverzeichnis
}%
{%
\providecommand{\myPath}{}% file exists=false: befinde mich im root_verzeichnis
}%
\input{\myPath_preamble/class_declaration_sa.tex}%% läd standalone-klasse mit tikz-argument
%//Tikzbibliotheken\\%
\usetikzlibrary{						% Bibliotheken zur direkten Einbindung in TIKZEdt
arrows, 
fit,
shapes.geometric, 
matrix,
calc,
decorations.markings,
decorations.pathreplacing,
decorations.pathmorphing,
backgrounds,
shadings, 
shadows,
positioning,
mindmap,
trees,
datavisualization
}%% diverse tikz-bibliotheken
%%Font etc.%%
\usepackage{lmodern}					% OK, Latin Modern font,											check: 25.08.18
\usepackage{xcolor} 					% OK, Definieren und Nutzen von versch. Farben						check: 25.08.18
\usepackage{graphicx}					% OK, Bereitstellen von \includegraphics							check: 25.08.18

%%Forest%%
\usepackage{forest}						% OK, Baumdarstellung aus dem linguistischen Bereich				check: 25.08.18
\useforestlibrary{edges}

%%Venndiagramm%%
\usepackage{venndiagram}				% OK, Definieren und Darstellen von Venndiagrammen					check: 25.08.18

%%Tabellen%%
\usepackage{booktabs}					% OK, Schönere Tabellen ohne vertikale Linien \toprule etc.			check: 25.08.18
\usepackage{tabularx}					% OK, Weiterer Spaltentyp passt Tabellenbreite  automatisch an		check: 25.08.18
\usepackage{multirow}					% OK, Zellenspannung über mehrere Zeilen							check: 25.08.18
%\usepackage{makecell}					% OK, Tabellenlayout (Tabaellenheader) ähnlich \multirow			check: 25.08.18
\usepackage{tablefootnote}				% OK, Fußnoten in Tabellen (\footnote funktioniert nicht)			check: 25.08.18
\usepackage{array} 						% OK, Erstellen eigener Columntypen in Tabellenumgebungen			check: 25.08.18

%%Grafiken und Plots%%
\usepackage{tikz} 						% OK, Natives zeichnen in Latex ,									check: 25.08.18
\usepackage{tikz-cd} 					% OK, Erstellen von kommutativen Diagrammen in Tikz,				check: 25.08.18
\usepackage{pgfplots}					% OK, Plotten von Daten,											check: 25.08.18
\pgfplotsset{compat=newest}				% OK, Einstellen der Kompatibilitätsversion,						check: 25.08.18
\usepackage{pgfplotstable}				% OK, Plotten und schreiben von Daten in Tabellen,					check: 25.08.18
\usepackage{pgfcalendar} 				% OK, Umrechnen von Datumskoordinaten,								check: 25.08.18
\usepgfplotslibrary{dateplot}			% OK, Plotten von Datumskoordinaten,								check: 25.08.18
\usepgfplotslibrary{units}				% OK, Darstellen von Einheiten als Achsenlabel,						check: 25.08.18%% nur laden wenn weitere graphic pakete benötigt werden (tabellen, pgfplot,...)
%%define tikz-stlyes here, colours etc.
\tikzset{
block_phantom/.style={block_normal, draw=red, fill=none},
block_phantom/.style={block_normal, draw=blue, fill=none}
}
%\tikzstyle{block_phantom}=[block_normal, draw=red, fill=none]%% tikz-styles, farben etc.
\begin{document}
\begin{forest}
for tree={grow=south},forked edges,
[Experimente zum Thema Kreativität in der Wirtschatfsinformatik
    [ Veröffentlichung
        [<1990]
        [<2000]
        [<2010]
        [>2010]
    ]
    [Aufgabenart
        [Generierung
            [Probandenrelevant]
            [Nicht Probandenrelevant]
        ]
        [Selektion
            [Probandenrelevant]
            [Nicht Probandenrelevant]
        ]
    ]
    [,phantom
        [,phantom
            [,phantom
                [test
                    [test1] [test2 [test3]]
                ]
            ]
        ]
    ]
    [Teilnehmermotivation
        [Bezahlung
            [>10 Euro]
            [<=10 Euro]
        ]
        [Leistungspunkte]
        [Teilnehmerrelevanz]
    ]
    [Kontrolle von Störgrößen
    [4
            [Gruppe6]
        ]
        [Gruppe5]
    ]
    [Gruppe 4
    [4
            [Gruppe6]
        ]
        [Gruppe5 [Gruppe4[Gruppe8[7[0[Gruppe4]]]]]]
    ]
    [Gruppe 5
    [Gruppe4
            [Gruppe6]
        ]
        [5]
    ]
    [Gruppe 6
    [Gruppe4
            [Gruppe6]
        ]
        [Gruppe5]
    ]
]
\end{forest}

\end{document}
%
      }
  \end{adjustbox}
\end{figure}
 
 
\begin{figure}[ht]
  \begin{adjustbox}{addcode={\begin{minipage}{\width}}{\caption{%
      Here is a caption of the figure which is so long that 
      it has to be wrapped over multiple lines, but should 
      not exceed the width (height after the rotation) of the image.
      }\end{minipage}},rotate=90,center}
      \includegraphics[width=\textheight]{_img/_forest/block_experiment.pdf}%
  \end{adjustbox}
\end{figure}

\input{\myPath_text/1_mainmatter/021_grundlagen_gm.tex}
\input{\myPath_text/1_mainmatter/021_grundlagen_krea.tex}
\input{\myPath_text/1_mainmatter/021_grundlagen_rev.tex}
\input{\myPath_text/1_mainmatter/021_grundlagen_wi.tex}
\input{\myPath_text/1_mainmatter/030_meta_rq1.tex}
\input{\myPath_text/1_mainmatter/031_literaturreview_identifikation.tex}
\input{\myPath_text/1_mainmatter/032_literaturreview_analyse.tex}
\input{\myPath_text/1_mainmatter/040_meta_rq2.tex}
\input{\myPath_text/1_mainmatter/041_ergebnisse_implikationen.tex}

% \printbibliography
%\input{text/anhang} % Anhänge
\end{onehalfspace} %Ende 1,5-facher Zeilenabstand


\end{document}
