%//Commands\\%
\newcommand{\fillOne}{green!35!black}						% Farbenneudefinition	
\newcommand{\fillTwo}{black!70}								% Farbenneudefinition	
\newcommand{\fillThree}{blue!30!}							% Farbenneudefinition	
\newcommand{\fillFour}{black!10!blue}						% Farbenneudefinition	

\newlength{\freewidth}										% Zur Berechnung noch freien Platzes bei prozentualer Angabe der Spaltenbreite in Tabellen

%//Settings\\%
\setcounter{secnumdepth}{3}									% Sektionstiefe definieren zur Darstellung im Inhaltsverzeichnis		
\setcounter{tocdepth}{4}									% Tiefe definieren zur Darstellung im Inhaltsverzeichnis

\renewcommand{\topfraction}{0.85} 							% Anteil einer Seite, die von Floats belegt sein darf (default 0.7)
\renewcommand{\textfraction}{0.1} 							% Anteil einer Seite, die mindestens Text sein muss (default 0.2)
\renewcommand{\floatpagefraction}{0.75} 					% Anteil einer Seite, die ein Float einnehmen darf, ohne auf die nächste Seite gesetzt zu werden (default 0.5)	

\newtheorem{theorem}{Satz}									% Theorem Satz
\newtheorem{definition}{Definition}							% Theorem Definition <-
\newtheorem{lemma}{Lemma}									% Theorem Lemma

\newcolumntype{L}[1]{>{\raggedright\arraybackslash}p{#1}} 	% Linksbündig mit Breitenangabe
\newcolumntype{C}[1]{>{\centering\arraybackslash}p{#1}} 	% Zentriert mit Breitenangabe
\newcolumntype{R}[1]{>{\raggedleft\arraybackslash}p{#1}} 	% Rechtsbündig mit Breitenangabe
%\newcolumntype{C}[1]{>{\centering\arraybackslash}m{#1}} 	% Vertikale zentrierung

\captionsetup[figure]{										% Captionsetup für Figure-Umgebung
%skip=5cm,
position=below,
justification=centering,
font=small
}
\captionsetup[table]{										% Captionsetup für Tabellen-Umgebung
%skip=5cm,
position=below,
justification=centering,
font=small}
\setlength\abovecaptionskip{10pt}
\setlength\belowcaptionskip{10pt}




