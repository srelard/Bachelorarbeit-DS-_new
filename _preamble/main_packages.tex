%//Allgemeine Pakete\\%
\usepackage[utf8]{inputenc}				% OK, Korrektes Encoding in Latex-Schrift, 							check: 25.08.18
\usepackage[ngerman]{babel}				% OK, Korrekte Schreibweise und Treenungsregelung					check: 25.08.18
\usepackage[T1]{fontenc}				% OK, Korrektes Font-Encoding			 							check: 25.08.18
\usepackage{amsmath} 					% OK, Mathematische Symbole und Schriften, 							check: 25.08.18
\usepackage{amsfonts} 					% OK, Mathematische Symbole und Schriften, 							check: 25.08.18
\usepackage{amssymb} 					% OK, Mathematische Symbole und Schriften, 							check: 25.08.18
\usepackage{amsthm} 					% OK, Typesetting Theoreme, 										check: 25.08.18
\usepackage{exscale}  					% OK, Skalierung von mathematischen Symbolen, 						check: 25.08.18
\usepackage{amstext} 					% OK, Text in mathematischer Umgebung, 								check: 25.08.18 [enthalten in amsmath]
\usepackage{verbatim} 					% OK, Funktion zum Schreiben von unformatiertem Text, 				check: 25.08.18
\usepackage{float} 						% OK, Ermöglicht das Erstellen von eigenen Float Umgebungen, 		check: 25.08.18
%\usepackage{makeidx}					% OK, Erstellen von Indizes zur Anlegung eines Glossars, 			check: 25.08.18
\usepackage{standalone} 				% OK, Kompilierung als eigenes Dok und Einbindung über Main,		check: 25.08.18
%\usepackage{import} 					% OK, Übernahme von relativen Verweisen in \inputs,					check: 25.08.18


%//Syntaktische Formatierung\\%
\usepackage{scrlayer-scrpage}			% OK, Definition und Manipulation von Kopf-, Fußzeile,				check: 25.08.18
\usepackage[onehalfspacing]{setspace}	% OK, Einstellen Zeilenabstand (onehalfspacing, doublespacing)		check: 25.08.18
\usepackage[							% OK, Definition des Seitenlayouts,									check: 25.08.18
includeheadfoot,						%
left=3cm,right=2cm,top=2cm,bottom=2cm,	%
showframe								%
]{geometry}								%
\usepackage[section]{placeins} 			% OK, Verhindert das Wandern von floats in eine andere section, 	check: 25.08.18
\usepackage{microtype} 					% OK, Abstand zwischen Zeichen anpassen zur Darstellung,			check: 25.08.18
\usepackage{lmodern}					% OK, Latin Modern font,											check: 25.08.18


%//Bilder und Tabellen\\%
\usepackage{xcolor} 					% OK, Definieren und Nutzen von versch. Farben						check: 25.08.18
\usepackage{graphicx}					% OK, Bereitstellen von \includegraphics							check: 25.08.18
\usepackage{array} 						% OK, Erstellen eigener Columntypen in Tabellenumgebungen			check: 25.08.18
\usepackage{rotating} 					% OK, Rotieren von floats in boxen (Tabellen, Bilder)				check: 25.08.18
\usepackage{adjustbox}					% OK, Ähnlich wie resizebox für normale Textumgebung,				check: 25.08.18
\usepackage{booktabs}					% OK, Schönere Tabellen ohne vertikale Linien \toprule etc.			check: 25.08.18
\usepackage{tabularx}					% OK, Weiterer Spaltentyp passt Tabellenbreite  automatisch an		check: 25.08.18
\usepackage{multirow}					% OK, Zellenspannung über mehrere Zeilen							check: 25.08.18
%\usepackage{makecell}					% OK, Tabellenlayout (Tabaellenheader) ähnlich \multirow			check: 25.08.18
\usepackage{tablefootnote}				% OK, Fußnoten in Tabellen (\footnote funktioniert nicht)			check: 25.08.18
\usepackage{subcaption} 				% OK, Darstellung von Tabellenumgebungen aus mehreren (auch Figure)	check: 25.08.18


%//Diverses\\%
\usepackage{paralist} 					% OK, Aufzählung wie \itemize \compactenum (nochmal anschauen)		check: 25.08.18
\usepackage[withpage]{acronym}			% OK, Genutze Abkürzungen werden das erste mal ausgeschrieben		check: 25.08.18
\usepackage{fancybox}					% OK, Verschiedene Varianten von \fbox zur Umrahmung von Text		check: 25.08.18
\usepackage[most]{tcolorbox}			% OK, Colorierte und umrahmte Textboxen wie \fbox für Theoreme		check: 25.08.18
\usepackage{layouts}					% OK, Darstellen von Dokumenteninformationen,						check: 25.08.18
%\usepackage[textsize=tiny]{todonotes}	% OK, Markierung und Erstellung von todo-Anmerkungen				check: 25.08.18
%\usepackage[timer]{regstats}			% Keine Funktion?, Erhebung von Statistiken (Kompiluierungszeit)	check: 25.08.18
%\usepackage[l2tabu, orthodox]{nag}		% Keine Funktion?, Prüfung nach veralteten Paketen					check: 25.08.18
%\usepackage{mylatexformat}


%//Theoreme\\%
%\usepackage{ntheorem} 					% OK, Darstellung von Theoremen										check: 25.08.18
%\usepackage{thm­tools} 				% OK, Darstellung von Theoremen <-									check: 25.08.18
\newtcbtheorem[number within=section, list inside=def]{mydef}{Definition}%
{enhanced, colback=white,colframe=\fillTwo,fonttitle=\bfseries, description delimiters={\flqq}{\frqq}, description font=\mdseries\ttfamily, boxrule=0.4pt,toprule=3pt, boxsep=3pt,arc=0mm, list text={test}, drop shadow}{def}


%//Zitate und Links\\%
\usepackage{csquotes}					% OK, Korrekte Darstellung von "" je nach Land						check: 25.08.18
\usepackage[]{hyperref}					% OK, Cross-Referencing und hypermarks im Dokument					check: 25.08.18
\hypersetup{
bookmarksopen=true,
%pdfpagelabels=true,					%Option `pdfpagelabels' has already been used,( hyperref) setting the option has no effect on input line 74
plainpages=false,
colorlinks=true,
citecolor=blue,
linkcolor=blue,
urlcolor=blue}

%//BibLatex\\%
\usepackage[backend=biber,				% OK, Neue Version von BibTex zur Darstellung Literaturverzeichnis	check: 25.08.18 
style=authoryear, %% Zitierstil
natbib=true, %% Bereitstellen von natbib-kompatiblen Zitierkommandos
hyperref=true, %% hyperref-Paket verwenden, um Links zu erstellen
maxcitenames=1,
uniquelist=false,
bibstyle=numeric,
url=false,
doi=false
]{biblatex}
%\addbibresource{bibfile.bib} 			% Einbinden der bib-Datei
%\DefineBibliographyStrings{ngerman}{andothers={et al.}} % et al. statt u.a.
%\renewcommand*{\nameyeardelim}{\addcomma\space} % Macht im Zitat ein Komma vor das Jahr
